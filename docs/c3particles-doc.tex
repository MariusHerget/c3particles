\documentclass[runningheads,a4paper]{llncs}

\usepackage{amssymb}
\setcounter{tocdepth}{3}
\usepackage{graphicx}
\usepackage{subfig}
%\linespread{2}

\usepackage{url}
\usepackage{csquotes}
\newcommand{\keywords}[1]{\par\addvspace\baselineskip
\noindent\keywordname\enspace\ignorespaces#1}

\begin{document}

\mainmatter  % start of an individual contribution

% first the title is needed
\title{c3particles: \\ Modeling Particle Systems in C++}

% a short form should be given in case it is too long for the running head
\titlerunning{c3particles}

%
\author{Rosalie Kletzander}
%
\authorrunning{c3particles}
% (feature abused for this document to repeat the title also on left hand pages)

\institute{Practical Course "Advanced Software Development with Modern C++"\\Summer Semester 2018\\MSc Tobias Fuchs\\Institute for Computer Science\\
Ludwig-Maximilians-Universit\"at M\"unchen\\
}

\maketitle


\begin{abstract}
Particle systems are used in many different areas, ..animation, simulation, research, ... .
 No matter the area of application, the basic rules governing these systems are the same: the laws of physics. C3particles (cpp particles) implements a model of a particle system in C++ that separates the physical concepts and laws from the underlying graphics library. This enables a mathematical formulation of the forces influencing the particles.

%keywords{network operating systems, programmable networks, Software-Defined Networking, SDN-controllers}
\end{abstract}


\section{Introduction}

%
%\textbf{
%\begin{figure}[here]
%\centering
%\includegraphics[width=0.5\textwidth]{images/nos-detailed-graph.png}
%\caption{Abstraction Models in the NOS}
%\label{fig:nos-abstractions}
%\end{figure}}





\bibliography{literature}
\bibliographystyle{plain}

\end{document}
